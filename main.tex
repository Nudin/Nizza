\documentclass[10pt,a4paper,twoside]{article}
\usepackage{a4wide}
\linespread{1.359140914229522617680} 
\usepackage[utf8]{inputenc}
\usepackage[german]{babel}
\usepackage{bibgerm}
\usepackage{amssymb}
\usepackage{graphicx}
\usepackage{hyperref}
\hypersetup{
%    bookmarks=true,
    pdftoolbar=true,
    pdfmenubar=true,
    pdffitwindow=false,
    pdfstartview={FitH},
    pdftitle={Das Nizza Modell},
    pdfauthor={Michael F. Sch\"onitzer},
    pdfsubject={Das Nizza Modell},
    pdfkeywords={Nizza-Modell},
    pdfnewwindow=true,
    colorlinks=true,
    linkcolor=black,
    citecolor=black,
    filecolor=black,
    urlcolor=blue
}

\author{Michael F. Schönitzer}
\title{Das Nizza Modell}

\newcommand{\refsec}[1]{siehe Kapitel \ref{#1}}
\newcommand{\degree}{$^\circ$}

\begin{document}
\maketitle

\section{Einleitung}
Im Jahr 2005 veröffentlichen Gomes, Levison, Morbidelli und Tsiganis drei Nature-Artikel\cite{Gomes2005}\cite{Tsiganis2005}\cite{Morbidelli2005} %Reihenfolge
in denen sie ein neues Modell für die Migration der Riesenplaneten unseres Sonnensystem vorstellten. Dieses Modell ist außerordentlich erfolgreich und wird Nizza-Modell genannt, da die Autoren zu besagter Zeit am Observatoire de la Côte d’Azur in Nizza arbeiteten.

Das Nizza-Modell betrachtet das Sonnensystem nachdem sich die Gasscheibe bereits aufgelöst hat, das Sonnensystem besteht nun aus der Sonne, den Planeten und einer Scheibe von Planetesimalen. Das Modell geht in Übereinstimmung mit Planetenentstehungsmodellen %wirklich?
davon aus, dass die Planeten damals nahezu kreisförmige, kompakte Orbits hatten.

Aus dieser Scheibe werden von den Planeten\footnote{Sofern nicht anderes angegeben, betrachten wir im folgenden immer nur die vier Gasriesen Jupiter, Saturn, Uranus \& Neptun. Die terrestrischen Planten sind aufgrund der geringen Masse vernachlässigbar.} immer wieder einzelne Planetesimale gestreut oder akkretirt. % Akkretiert? 
Dabei kommt es durch die Impulsübertrag zu einer Änderung der Planetenorbits\cite{Tsiganis2005},
wodurch Saturn, Uranus und Neptun langsam nach außen wandern, während Jupiter langsam nach innen wandert\cite{Tsiganis2005}\cite{Hahn1999}. % Hahn nicht gelesen
Mit der Zeit kommen sich die Planeten durch die Migration näher und es kommt zu einer Resonanz,
welche zu einer schlagartigen Erhöhung der Exzentrizitäten der Umlaufbahnen von Jupiter und Saturn, auf Werte die mit den heutigen vergleichbar sind, führt.
Die Planeten Saturn Uranus und Neptun kommen sich gegenseitig und der Scheibe aus Planetesimalen nahe. Dadurch werden die Planetesimale praktisch schlagartig zerstreut, das System wird instabil und ein Teil der Planetesimale fliegt in das innere Planetensystem und löst dort das Late Heavy Bombardment aus.

Nach etwa hundert Millionen Jahren erreichen die Planeten schließlich ihre heutigen Entfernungen, ihre Exzentizitäten werden gedämpft und das System stabilisiert sich wieder. Die übriggebliebenen Planetesimale bilden die heutigen transneptunischen Objekte.

Neben den Plantenorbits (\refsec{Orbits}), dem Late Heavy Bombardment (\refsec{LHB}) und dem Kuipergürtel (\refsec{Kuiper}), kann das Modell, wie wir sehen werden, auch die Trojaner (\refsec{Trojaner}), die irregulären Monde (\refsec{Monde}) und die Achse des Uranus (\refsec{Uranusachse}) zumindest prinzipiell erklären.
Abschließend werden wir uns mit einer möglichen Erweiterung des Nizzamodells um zusätzliche Anfangsplaneten beschäftigen (\refsec{mehrPlaneten}).

\section{Das Nizza-Modell}\label{Orbits}
In diesem Kapitel werden wir das Nizzamodell wie in der ursprünglichen Arbeit\cite{Tsiganis2005} beschrieben erläutern.

In der Simulation berücksichtigten die Forscher dabei die vier Riesenplaneten, sowie eine Scheibe aus Planetesimalen im Gravitationsfeld der Sonne. % Bäh: Forscher?
Dies ist ein plausibler Ausgangszustand für die Situation nachdem der Sonnenwind die Gaswolke %Sonnenwind, Gaswolke?
zerstreut hat. % bäh: formulierung
Als Anfangsparameter wählten Tsiganis et. al in den Simulationen für die große Halbachse des Jupiterorbits $a_J = 5,45 AU$. Saturn wurde wenige Zehntel AU von der bei $a_{1:2} = 8,65 AU$ gelegenen 1:2-MMR gesetzt\cite{Tsiganis2005}. Die Orbitalverteilung der transneptunischen Objekte lässt darauf schließen, dass Neptuns Migration innerhalb von 20 AU begonnen hat\cite{Tsiganis2005}. % F***: die erklärung von MMRs ist erst weiter unten
Die beiden Eisplaneten wurden in den Simulationen von der Gruppe von Tsigani mit großen Halbachsen von 11-13 AU respektive 13,5-17 AU, mit einem Mindestabstand von 2 AU gestartet.
Die Planetenorbits waren, wie von den Planetenentstehungsmodellen gefordert, % formulierung, Quelle
nahezu kreisförmig und coplanar ($e, i \approx 10^{-3}$).

Die Planetisimalenscheibe – ein Überbleibsel der Planetenentstehung – begann direkt hinter dem Orbit des zweitern Eisplaneten und reichte bis zu einer Entfernung von 30-35 AU\cite{Tsiganis2005} hinaus. Sie bestand aus 1000-5000 gleich schweren Brocken, mit einer Gesamtmasse von 30 bis 50 Erdmassen. %Ausdehnung steht oben, runter?
Die Flächendichte der Scheibe wurde in den Simulationen linear mit dem Sonnenabstand abfallend gewählt. In einigen Simulationen wurde eine dynamisch "heiße"\ Scheibe in anderem eine "dynamisch kalte"\ Scheibe gewählt, wobei mit dynamisch heiß gemeint ist, dass Exzentrizität und Inklination relativ groß waren: $e \approx \sin i \approx 0.05 $, im Vergleich zu $e \approx \sin i \approx 10^{-3} $ im kalten Fall.\cite{Tsiganis2005}
Die Eigengravitation der Scheibe wurde in den Simulationen ignoriert\cite{Tsiganis2005}.
Die Simulationen wurden mit zwei verschiedenen symplektischen N-Körper Codecs durchgeführt: MERCURY von John Chambers und SyMBA, einer Erweiterung des von Hal Levison und Martin Duncan entwickeltem SWIFT.
% http://www.boulder.swri.edu/~hal/swift.html
% http://www.arm.ac.uk/~jec/
Der Zeitschritt betrug zwischen einem viertel und einem halben Jahr.\cite{Tsiganis2005} % REALY?

Aus dieser Scheibe werden von den Planeten immer wieder einzelne Planetesimale gestreut oder akkretirt. Dabei kommt es durch die Impulsübertrag zu einer Änderung der Planetenorbits\cite{Tsiganis2005}. %Kinderkacke
Die Simulationen zeigen, dass Saturn, Uranus und Neptun langsam nach außen wandern, während Jupiter langsam nach innen wandert\cite{Tsiganis2005}\cite{Hahn1999}. % Hahn nicht gelesen

Während der Migration werden die Exzentrizitäten durch den gravitativen Einfluss der Planetesimale, bekannt als dynamische Reibung, gedämpft.\cite{Tsiganis2005}
Mit der Zeit kommen die Planeten sich durch die Migration näher, so dass Ressonanzen (englisch mean motion resonance, MMR) auftreten.
Man betrachtet im Nizza-Modell die 2:1-Resonanz zwischen Jupiter und Staturn, welche nach ein paar hundert Millionen Jahren auftritt.
Bei einer MMR handelt es sich um den Fall, dass die Umlaufdauern zweier Planeten im Verhältnis zweier kleiner natürlicher Zahlen steht. In der hier relevanten 2:1-MMR zwischen Jupiter und Saturn, läuft also Jupiter genau zweimal um die Sonne, während Saturn in der selben Zeit genau einmal seinen Orbit durchlauft, die Planeten "begegnen"\ sich also immer am selben Ort. %Nochmal Prüfen
Die 2:1-Resonanz zwischen Jupiter und Saturn ist die stärkste Resonanz. % einbauen
Die Autoren testeten auch Anfangsbedingungen, welche zum Auftreten von anderen MMRs führten. Doch weder die 2:3 MMR zwischen Saturn und dem inneren Eisriesen, noch die 1:2 MMR zwischen den beiden Eisriesen waren stark genug um die Bahn von Jupiter zu beeinflussen\cite{Tsiganis2005}. % Tsiganis2005 hats noch genauer
Die Resonanz für zu einer schlagartigen Erhöhung der Exzentrizitäten der Umlaufbahnen von Jupiter und Saturn, auf Werte die mit den heutigen vergleichbar sind. %genauer erklären?
Dadurch stören Jupiter und Saturn die Eisplanenten, so dass auch deren Exzentrizitäten abhängig von den genauen Anfangsparametern (Massen und großen Halbachsen der Planeten) mehr oder weniger stark anwachsen\cite{Tsiganis2005}.
Da die Planetenorbits sehr dicht aneinander liegen, führen die hohen Exzentrizitäten zu überschneidenden Bahnen\cite{Tsiganis2005}, es kommt dadurch zu Begegnungen von Planeten. %Bäh: Begegnungen
Dies hat wiederum zwei Effekte: die Inklinationen der Planeten wächst um $1^\circ-7^\circ$ und die Eisriesen werden hinaus in die Planetesimalscheibe gestreut,
wodurch nun schlagartig eine große Menge an Planetesimalen ins Innere gestreut werden und sich folglich die Migrationsrate drastisch erhöht\cite{Tsiganis2005}.
Die große Anzahl an kleinen Objekten im Bereich der Planetenbahnen, führt jedoch zu dynamischer Reibung, wodurch die Exzentrizitäten und Inklinationen wieder langsam sinken und sich das System somit wieder stabilisiert\cite{Tsiganis2005}.
Wenn die Planetesimalscheibe fast vollständig zerstreut ist, stoppt die Migration der Planeten und sie erreichen ihre endgültigen Bahnen\cite{Tsiganis2005}.

\newcommand{\AU}{\;\mathrm{AU}}
\newcommand{\DII}{\Delta a_{I_1,I_2}}
\newcommand{\DSI}{\Delta a_{S,I_1}} % Formeln, nicht kursiv
Diese hängen von dem Verhalten des Systems zum Zeitpunkt direkt nach der Resonanz ab. In den 43 Simulationen in \cite{Tsiganis2005} zeigte sich, dass von den zahlreichen Anfangsparametern der Abstand der Eisriesen $\DII$ und vor allem der Abstand zwischen Saturn und dem inneren Eisriesen $\DSI$ die größten Auswirkungen haben\cite{Tsiganis2005}. %Diese spezifizieren?
Wie oben schon erwähnt, wurden für $\DII$ Werte zwischen etwa zwei und sechs AU verwendet, während die Werte von $\DSI$ zwischen ~2.5 und ~5 AU betrugen\cite{Tsiganis2005}.
Wählt man den Abstand zwischen Saturn und dem ersten Eisplaneten klein, so steigt die Wahrscheinlichkeit, dass einer der Eisplaneten von Saturn auf ein die Jupiterbahn kreuzendes Orbit gestreut wird und dann von Jupiter aus dem System geschleudert wird. Für $\DSI \le 3 \AU $ geschah dies in 14 der 43 Simulationen (33\%).
Wählt man den Abstand hingegen sehr groß ($\DSI \approx 5 \AU$), so ist das System möglicherweise nicht mehr kompakt genug, so dass überhaupt keine Begegnungen von Planeten stattfinden. %bäh: Begegnungen

Für die restlichen Fälle, welche allesamt nach einer wie oben beschriebenen Migrationsphase wieder zur Ruhe kommen %bäh: zu Ruhe kommen
gilt, dass es für Abstände $\DII \ge 3.5 \AU$ zwar zu Interaktionen der beiden Eisplaneten unter einander, nicht jedoch zwischen einem Eisriesen und Saturn kommt, während es in Simulationen mit kleinerem $\DII$ auch zu solchen kommt. In Übereinstimmung mit der Originalveröffentlichung, bezeichnen wir erstere Art von Simulationen als Klasse A und letztere als Klasse B\footnote{Die beiden Fälle in der es zu keiner Begegnung kam, wurden aus mir nicht ganz nachvollziehbaren Gründen nicht ausgeschlossen sondern zur Klasse A gerechnet.}.

Wechselwirkungen mit Jupiter führen bei Klasse B dazu, dass die Eisriesen trotz dynamischer Reibung ihre Exzentrizität nicht verlieren und die Phase der Schnellen Migration dauert bei diesem Typ kürzer.
14 der Simulationen waren vom Typ B, 15 vom Typ A.
Die Durchschnittswerte und Standardabweichungen für die großen Halbachsen, Exzentrizitäten und Inklinationen der Planeten von beiden Gruppen sind in Grafik \ref{AvsB} aufgetragen. Wie man sieht stimmen die Resultate beider Klassen fast mit den tatsächlich beobachten Werten überein, im Fall der Klasse B ist die Übereinstimmung jedoch deutlich größer – hier liegen sogar alle Messewerte innerhalb nur einer Standardabweichung um den Mittelpunkt\cite{Tsiganis2005}.
Dies stellt einen bedeutenden Erfolg des Modells dar und hebt sich von vorhergehenden Modellen ab.
So hatte zum Beispiel das Vorgängermodell von Gomes, Morbidelli und Levinson \cite{Gomes2004} %Namen prüfen
das Problem zwar vorsagen zu können, dass Neptun eine große Halbachse von etwa 30 AU hat, jedoch wart die Uranusbahn dabei zu dicht an der Sonne.
Die großen Halbachsen der Eisriesen für die Klasse B betragen im Nizza-Modell hingegen $a_U = 19,3 \pm 1,3 \AU$ und $a_N = 29,9 \pm 2,4 \AU$, was mit den tatsächlichen Werten von $a_U = 19,2 \AU$ und $a_N = 30,1 \AU$ sehr gut übereinstimmt\cite{Tsiganis2005}.

Der finale Abstand zwischen Jupiter und Saturn hängt vor allem von der Masse der Materie %bäh: finale
die sie während der instabilen Phase streuen ab, welche wiederum von der Masse der Planetenscheibe abhängt\cite{Tsiganis2005}. % richtig?
Höhere Massen der Scheibe führen zwar zu stabileren Endsystemen, wird die Masse jedoch größer als $\approx (35-40) M_E$ wird der Abstand zwischen Jupiter und Saturn im fertigsimuliertem System zu groß\cite{Tsiganis2005} und wählt man eine Masse von $\approx 50 \AU$, so tritt zwischen Jupiter und Saturn zusätzlich auch eine 2:5 Resonanz auf und die dynamische Reibung wird so groß, dass Exzentrizitäten kleiner als in Realität ausfallen\cite{Tsiganis2005}.

Die dynamische Temperatur der Scheibe wirkt sich auf die Exzentrizitäten aus: Heißere Scheibe führen zu höheren Exzentrizitäten von Jupiter und Saturn. Die beobachtete Existenz von zahlreichen plutogroßen Transneptunischen Objekten spricht laut den Autoren für eine Scheibe mit der Dynamik der heißeren der beiden getesteten Scheiben\cite{Tsiganis2005}. % Tsiganis2005 zitiert hier Stern 1991

%Gemäß aktueller Planetenentstehungsmodelle sollten sich die Gasplaneten auf kreisförmigen coplanaren Orbits gebildet haben, jedoch 
%warum ein Migrationsmodell überhaupt von nöten ist.
%Darüberhinausgehend nimmt das Modell an, dass bei der Planetenentstehung eine Scheibe von Planetesimalen entstand, die von außerhalb der Planetenorbits bis hinaus zu einer Entfernung von 30-35 AU\cite{Tsiganis2005} reichte und eine Gesamtmasse von etwa 35 Erdmassen hatte. %Überprüfen, Quellen, Doppelung entfernen


\section{Late Heavy Bombardment}\label{LHB}
Geologische Untersuchungen von Mondgestein zeigen, dass es 700 Millionen Jahre nach der Plantenentstehung
eine enorme Häufung von Einschlägen gab, man nenn dieses Ereignis das Late Heavy Bombardment (LHB) oder auf deutsch auch das Große Bombardement.
%% Genauere Erklärung, bessere Quelle!
Die Planetenentstehungsmodelle können eine solche Häufung zu einem so spätem Zeitpunkt nicht erklären.
Es gibt einige Modelle zur Erklärung des LHBs jedoch sind diese relativ gekünstelt und kein Notwendiger Bestandteil der ansonsten abgeschlossenen Beschreibung der Entwicklung des Sonnensystems.\cite{Gomes2005} %Andre Quelle, genauer?
Im Rahmen des Nizza-Modells liegt es natürlich nahe anzunehmen das LHB sei durch die während der Instabilitätsphase zerstreuten Teilchen der Planetesimalenscheibe entstanden. % bäh: Teilchen

Das die Planetesimalenscheibe aufgrund ihrer Sonnenentstehung vermutlich eisig war und sie wie wir in Kapitel \ref{Kuiper} sehen werden, der Ursprung der heutigen Transneptunisch Objekte war, kann man sie als Kometen bezeichnen\cite{Gomes2005}. % wohin damit??

%Bei folgenden Abschnitten bin ich mir noch nicht ganz sicher.
In anderen, ähnlichen Modellen – einschließlich frühen Versionen des Nizza-Modells – hatte man das Problem, dass die schnelle, durch das Streuen von Planetesimalen bedingte, Migration unmittelbar nach Simulationsbeginn eintrat und man somit nicht erklären konnte warum das LHB erst nach etwa 700 Myr stattfand.
Dies lag jedoch an einer falschen Startbedingung: Man hatte hierbei die Planetesimalen in die direkte Umgebung zu den Planeten gesetzt, was zwar zu einer gewünschten Migration führt, jedoch eine unnatürliche Startsituation ist,
denn es ist davon auszugehen, das die Migration durch Interaktion mit den Planetesimalen zum Zeitpunkt als das System noch in der Gasscheibe eingebettet war, im Vergleich zur Migration durch Interaktion mit der Gasscheibe vernachlässigbar ist.
Deshalb sollte die Anfangsbedingung des Nizzamodells gerade der Zustand sein, in welchem sich das Sonnensystem direkt nach dem Auflösen der Gasscheibe befand. Betrachten wir also nur die Migration nach dem Auflösen der Gasscheibe, so sollten alle zum Startzeitpunkt existierenden Planetesimale sich auf Orbits befinden, die stabil genug sind um bis zu diesem Zeitpunkt überhaupt überlebt zu haben. %Doppelung, am Anfang des Satzes zum Ende des letzten
Die Planetesimalscheibe sollte also nur Teilchen auf Bahnen haben deren dynamische Lebenszeit größer als die Lebensdauer der Gasscheibe ist.\cite{Gomes2005} Berücksichtigt man diese Bedingung, erhält man die schon in Kapitel \ref{Orbits} vorgegriffene Anfangsbedingung, das die Planetesimalscheibe hinter dem letzten Planeten beginnt, und zwar hinter etwa 15,3 AU.\cite{Gomes2005}

Die anfängliche Migrationsrate hängt nun von der Rate ab, mit welcher Planetesimale auf Planetenorbits schneidende Orbits gestreut werden. Der Zeitdauer bis zum auftreten der MMR hängt schließlich von den folgenden Parametern ab:
1. Dem Anfangsabstand zur Ressonanz, 2. der Dichte der Scheibe am inneren Rand und 3. der Abstand zwischen dem inneren Rand der Scheibe und dem äußeren Eisplaneten.\cite{Gomes2005}

In acht Simulationen untersuchten sie die Abhängigkeit des Ressonanzzeitpunktes in Abhängigkeit von der Position des inneren Rands der Planetesimalscheibe, bei ansonsten gleichen Parametern – inklusive gleicher Scheibenmasse und Dichte. Wie in Bild \ref{Bildfehlt} zu sehen ist, steigt die Zeit stark mit der Entfernung des inneren Rands an. Für große Werte $\gtrsim 15,3 \AU$ wie wir sie oben erhalten haben, tritt die Resonanz 192 bis 880 Millionen Jahre nach beginn der Simulation auf, was zum Zeitpunkt des LHB passen würde.\cite{Gomes2005} % Achtung: nach beginn der Simulation, nicht nach Planetenentstehung wie oben.
Durch zusätzliche Variation anderer Parameter konnte das Eintreten der Resonanz auf bis zu 1,1 Milliarden Jahre\cite{Gomes2005} verzögert werden – somit ist klar, dass das später Auftreten des großen Bombardements bei geeigneter Parameterwahl kein Problem bei Erklärung durch das Nizzamodell ist.

Als zweite wichtige Überprüfung, schauten sich die Wissenschaftler die Masse des auf den Mond treffenden Materials an. Bei den Simulationen waren dies etwa $9 \cdot 10^{21} \mathrm{g}$, davon trafen 50\% den Mond in den ersten 3,7 Millionen Jahren, 90\% in den ersten 29 Millionen Jahren. Das Bombardement fand demnach während einer relativ kurzen Zeitspanne statt. Die dabei durchschnittlich auf dem Mond treffende Gesamtmasse wurde auf $\left(8,4 \pm 0,3\right) \cdot 10^{21} \mathrm{g}$ bestimmt\cite{Gomes2005}.

Hinzukommt, dass es bei der Migration von Jupiter und Saturn von ihrer Position in der 2:1-MMR zu ihren heutigen Orbits zu einer weiteren Art von Resonanzen – die sogenannten Secular resonances treten auf wenn die orbitale Präzession zweiter Körper synchronisiert sind.\footnote{Ein deutscher Name ist mir nicht bekannt.} – % besser erklären was das ist!
kommt, welche über den Asteroidengürtel streifen. % bäh: streifen
Durch diese können Asteroiden Orbits mit so großen Exzentrizitäten und Inklinationen erhalten, so das diese bis ins innere Sonnensystem reichen. Wir müssen also die Masse der Asteroiden bestimmen, welche zusätzlich zu den Planetesimalen % die man als Kometen betrachten kann… <– hier oder wo anders?
mit dem Mond kollidieren.
Dafür führten die Autoren weitere numerische Simulationen durch in welchen sie die Auswirkungen von Sonne, Venus, Erde, Mars, Jupiter und Saturn auf den durch 1.000 masselose Teilchen gebildeten Asteroidengürtel untersuchten.
Die Migration der Gasriesen wurde dabei, durch das Hinzufügen von geeigneten Termen, künstlich hinzugefügt. % formulierung
Die große Halbachsen der Asteroiden betrugen dabei zwischen $2$ und $3,5 \AU$ und hatten Exzentrizitäten zwischen 0 und 0,3, sowie Inklinationen zwischen 0\degree und 30\degree – wobei das Perihel immer größer als $1,8 \AU$ und das Aphel immer kleiner als $4 \AU$ gewählt wurde. % Grund erläutern
Als Migrationsratten wurden in den Simulationen unterschiedliche Resultate der ursprünglichen Simulationen \refsec{Orbits} gewählt. %formulierung?

Gomes et al.\cite{Gomes2005} unterscheiden zwei Wege, auf welchen ein Asteroid auf ein die Erdbahn kreuzendes Orbit kommen kann. (1) Entweder es kommt zu einer Secular resonance zwischen der Periheldrehung des Körpers und der Periheldrehung Saturns, wodurch die Exzentrizitäten des Asteroiden steigen und er auf eine Erdbahnkreuzende Bahn gelangt, oder (2) sie werden im Asteroidengürtel dynamisch angeregt und wandern dann langsam aus dem Asteroidengürtel heraus\cite{Gomes2005}.
Weg (1) ist schneller, so das 50\% der Einschläge dieser Art bereits in den ersten 10 Myr (90\% in 30 Myr) stattfindet, während die Einschläge durch, auf letztere Art auf Kollisionskurs gebrachte, Asteroiden innerhalb von 50 Myr zu 50\% stattfanden (90\% in 150 Myr)\cite{Gomes2005}.
In den Simulationen von Gomes et al. war der zweite Typ häufiger, wobei dies wenig aussagt, da
die Häufigkeitsverteilung der beiden Wege vermutlich stark von den genauen Bedingungen abhängt\cite{Gomes2005} %genauer?
Die Masse an Asteroiden die den Mond gemäß diesen Simulationen trifft wurde auf $\left( 3-8 \right) \cdot 10^{21} \mathrm{g}$, % anderer strich statt minus? space
also einen ähnlich hohen Wert wie für die Planetesimale bestimmt – wobei die Unsicherheiten bei der Bestimmung der Masse der Asteroiden zu groß ist um ein Verhältnis zwischen den beiden Komponenten anzugeben. Auch ist noch nicht genau untersucht worden, inwieweit dieses Verhältnis eine Funktion von der Zeit und der Teilchengröße ist – so waren in diesen Simulationen die Planetesimale in den ersten 30 Millionen Jahren dominierend, die Asteroiden länger auf den Mond niederprasselten\cite{Gomes2005}.

Die Dauer des LHB beträgt in diesem Modell also zwischen 10 und 150 Millionen Jahren – eine genauere Bestimmung ist nicht möglich, da die Verhältnisse der beiden durch Asteroiden beigetragenen und des durch Planetesmiale beigetragenen Anteile sensibel von den genauen Anfangsparameter – insbesondere der genauen anfänglichen Struktur des Asteroidengültels – abhängen und daher nicht genauer bestimmt werden können\cite{Gomes2005}.

Vergleichen wir nun die Vorhersagen des Modells mit bekannten Messgrößen. Das Nizzamodell kann den für andere Modelle problematischen späten Zeitpunkt des Bombardements erklären. Auch die Menge des dabei auf den Mond eingebrachten Materials passt größenordnungsmäßig gut: So wurde aus der Anzahl und Größe der Mondkrater eine ungefähre Masse von $6 \cdot 10^{21} \mathrm{g}$ bestimmt, während die Simulationen $\left(8,4 \pm 0,3\right) \cdot 10^{21} \mathrm{g}$ an Planetesimalen und weitere $\left( 3-8 \right) \cdot 10^{21} \mathrm{g}$ an Asteroiden ergab.% [4] aus Gommes2005 zitieren!?
Das Modell sagt wie schön in Bild \ref{Bild_fehlt} zu sehen einen sehr plötzlichen Beginn des Bombardements voraus leider sind die bisher gemessenen Daten noch zu schlecht um dies zu überprüfen.
Das Nizza-Modell sagt voraus, dass sowohl Asteroiden, also auch Planetesimale auf dem Mond eingeschlagen sind, dies stimmt damit überein, dass kosmochemischen Analysen davon ausgehen, dass einige der Mondkrater durch Einschläge von Asteroiden geformt wurden.
Die Streuung von Asteroiden führt auch dazu, dass der Asteroidengürtel um einen Faktor von ungefähr 10 ausgedünnt wird, was zu bestehenden Modellen passt.  %Streuung Genauer!, Modelle citen!
Die Menge an Kometen, die während des LHB auf die Erde trafen ergibt sich im Nizzamodell zu $\approx 1,8 \cdot 10^{23} \mathrm{g}$ was etwa 6\% der heutigen Ozeanmasse beträgt und somit kompatibel zu der durch Isotopenhäufigkeit (D-zu-H-Verhältnis) bestimmten Obergrenze der durch Kometen eingebrachten Wassermenge\cite{Gomes2005}. % cite [13] in Gomes2005?!

Abschließend möchte ich noch darauf hinweisen, dass es neuere geologische Forschungen gibt, welche die Notwendigkeit der Existenz des Late Heavy Bombardments in Frage stellen\cite{Spudis2011}. % ausführlicher


\section{Trojaner}\label{Trojaner}
foobar

\section{reguläre und irreguläre Monde}\label{Monde}
\subsection{irreguläre Monde}
Als irregulären Mond bezeichnet man einen natürlichen Satelliten auf Orbits die weit vom Planeten entfernt sind und eine starke Inklination haben\cite{Nesvorny2007}. %bessere Quelle
Häufig haben Sie auch retrograde Umlaufbahnen und große Exzentrizitäten.
Inzwischen sind über 90 irreguläre Monde bei den Gasplaneten bekannt\cite{Nesvorny2007}\footnote{Andere Quelle sprechen von 113}. %bessere Zahl & Quelle
Während sich reguläre Satelliten unseres Wissens nach durch Akkretion aus der planetaren Gasscheibe gebildet haben, ist die Herkunft von irregulären Monden noch ungeklärt.
Sie müssen auf irgendeine Art von dem Planeten aus einem heliozentischem Orbit eingefangen sein worden, da eine Entstehung aus der planetaren Gasscheibe durch Akkretion wie bei den regulären Monden ist aus mehreren Gründen nicht möglich:
Sie sind von den regulären Trabanten räumlich zu stark getrennt um aus der selben Gasscheibe entstanden zu sein, per Akkretion können keine derartig großen Exzentrizitäten entstehen und vor allem können aus einer Gasscheibe keine retrogard umlaufende Planeten entstehen\cite{Nesvorny2007}. % besser formulieren!

Das System Sonne-Planet-einzufangender Körper reicht jedoch nicht aus, da das System Zeitreversibel ist und somit jeder Weg des Körpers von der heliozentischen Bahn zu der Planetenumlaufbahn auch wieder ein möglicher Weg zurück ist\cite{Nesvorny2007}. % formulierung
Es braucht also einen Mechanismus auf welche Weise ein Trabant dauerhaft eingefangen werden kann. Dazu wurde unter anderem vorgeschlagen, dass der Körper Energie durch den Reibung an den Planeten umgebenden Gas verliert. % die anderen 2 Modelle erwähnen?
Dieses Modell kann zwar einige der irregulären Moden von Jupiter erklären, nicht jedoch alle, da einige weiter entfernt als die Gasscheibe gereicht hat und auch auf die irregulären Monde von Neptun und Uranus lässt sich dieses Modell vermutlich nicht anwenden. % genauer?
Migrationsmodelle, wie das Nizzamodell führen zu einem weiteren großen Problem für derartige "`gas-drag"'-Modelle. Kommt den so entstandenen irregulären Monden ein großer Planetesimal oder ein Planet zu nahe, werden sie äußerst effizient wieder aus dem System des Planeten gefegt\cite{Nesvorny2007}. Da im Nizzamodell sich die Planeten Saturn, Uranus und Neptun entsprechend nahe kommen, können die irregulären Monde die wir heute um diese Planeten sehen somit nicht aus der Zeit vor der Migration stammen\cite{Tsiganis2005}\cite{Nesvorny2007}, was für "`gas-drag"'-Modelle aber nötig wäre, da zum Zeitpunkt der Migration die Gasscheibe ja sich bereits aufgelöst hat\footnote{Es ist natürlich möglich, dass es früher andere "`Generationen"' von irregulären Monden gab, welche durch den Energieverlust in der Gasscheibe entstanden sind. }. % "`gas-drag"'-Modelle eindeutschen, definieren oder ohne Anführungszeichen?; pfuibäh: Fußnote wird derzeit über zwei Seiten verteilt

Eine weitere wichtige Eigenschaft der irregulären Monde sind ihre vielfältig variierten und keinem klaren Gradienten folgenden Farben. Würden die Monde aus lokalen Umgebung der Planeten stammen, müssten alle Monde eines Planeten eine ähnliche Farbe haben und die Farben der Monde müsst vom Sonnenabstand des Planeten abhängen\cite{Nesvorny2007}. Das beides nicht der Fall ist, ist ein Zeichen dafür, dass sie ein Mix aus Planetesimalen von ursprünglich ursprünglichen Orten sind. % formulierung, Quellen

Ein erster Ansatz um irreguläre Satelliten im Rahmen des Nizza-Modells zu erklären wurde 2006 von Ćuz und Gladman vorgestellt\cite{Cuk2006}. Dieser geht davon aus, dass irreguläre Monde auf relativ kleinen Bahnen durch "`gas-drag"'-Modelle entstehen und erst später ihre Bahnen durch die Instabilität des Systems anwächst. Dies konnte die oben angesprochenen Probleme der "`gas-drag"'-Modelle mit weit entfernten irregulären Monden zumindest bei Saturn – eingeschränkt auch bei Jupiter und Uranus – beheben\cite{Cuk2006}. % ausführlicher!?
Dieses Modell berücksichtigt jedoch nicht, dass es wie oben erwähnt im Nizza-Modell zu Annäherungen zwischen Planeten kommt, welche derartige irregulären Monde wieder aus ihrer Umlaufbahn werfen würde. Auch die anderen Erkenntnisse legen es nahe, dass die Monde erst während der Instabilitätsphase durch das Einfangen einiger der zahlreichen durch das Sonnensystem fliegenden Planetesimale entstanden. % bäh: fliegen, entstanden

Als Einfangmechanismus dient hierbei ein dritter massiver Körper,
% es fehlt die exchange reaction – rein, oder nicht? wenn ja, wo?
welcher sich dem Planeten auf weniger als einen Hillradiuses nähert und Körper, % Hillradius muss noch definiert und erklärt werden, vermutlich oben bei den regulären Monden.
welche zeitgleich ebenfalls durch den Hillradius fliegen anregen, so dass einige von ihnen auf stabilen Orbits enden. % bäh: enden
Da sich die Gasplaneten nach dem Eintreten der Resonanz näher kommen, während gleichzeitig eine große Anzahl an Planetesimalen das Sonnensystem durchstreift,
ergibt sich folgendes Bild: zwei Planeten kommen sich so nahe, dass Sie gegenseitig in den Hillradius des anderen eindringen,
einige der ebenfalls durch den Hillradius der Planeten fliegenden Planetesimale wird dadurch auf weit entfernten Bahnen um einen der Planeten mit großen Inklinationen eingefangen und % große Inklinationen?
kreisen nun als irreguläre Monde um den Planeten\cite{Nesvorny2007}.
% wohin damit:
In der Ausklingzeit des Nizza-Modells, als die Exzentrizitäten durch dynamische Reibung gedämpft werden und die Scheibe sich langsam leert, ist es möglich, dass die Bahnen der irregulären Monde durch vorbeifliegende besonders massive (vergleichbar mit Planeten) Planetesmiale können die Bahnen noch einmal geändert werden. % bäh: die Klammer; Bahnen -> veralgm.??
Dies ist durchaus realistisch, da es keinen Grund gibt, warum sich außer den vier Riesenplaneten kein anderen großen Objekte gebildet haben sollen\cite{Nesvorny2007}.
%



%%%% Nizza-Modell einheitlich scheiben


\newpage % ENTFERNEN, bzw. VERSCHIEBEN sobald untere Kapitel vorhanden sind.

\section{Kuipergürtel}\label{Kuiper}
%Asteroidengürtel

\section{Uranusachse}\label{Uranusachse}

\section{4/5/6 Planeten}\label{mehrPlaneten}

\section{Zusammenfassung}

\newpage
\renewcommand{\thesection}{\Alph{section}}
\setcounter{section}{0} % Problem, Kapitel 1 und zwei sind doppelt im PDF-Verzeichniss
\section{Eigene Berechnungen}
\section{Literaturverzeichnis}

\bibliography{literatur}{}
\bibliographystyle{gerplain}

\end{document}
